\usepackage{ifthen}
\usetikzlibrary{calc}
\usetikzlibrary{positioning}
\usetikzlibrary{shapes.geometric}
\usetikzlibrary{shapes.misc, decorations.pathreplacing}
\usetikzlibrary{arrows.meta}

\makeatletter

\pgfmathdeclarefunction*{binom}{2}{%
    \begingroup
        \pgfmathcontinuelooptrue
        \c@pgf@counta=0
        \c@pgf@countb=1
        \c@pgf@countc=1
        \loop
        \ifpgfmathcontinueloop
            \ifnum\c@pgf@counta=#2\relax
                \pgfmathcontinueloopfalse
            \else
                \c@pgf@countd=#1
                \advance\c@pgf@countd by-\c@pgf@counta
                \multiply\c@pgf@countc by\c@pgf@countd
                \divide\c@pgf@countc by\c@pgf@countb
                \advance\c@pgf@counta by1\relax
                \advance\c@pgf@countb by1\relax
            \fi
        \repeat
        \pgfmathparse{int(\c@pgf@countc)}%
        \pgfmath@smuggleone\pgfmathresult
    \endgroup
}

\newcommand\intersect[4]{
    % Computes whether two given sets intersect, returns 1 if they intersect, 0 otherwise
    % #1 = variable to store the result
    % #2 = length
    % #3 = number 1 (base 10)
    % #4 = number 2 (base 10)
    \def\@res{0}

    \pgfmathsetmacro\@tmp{#3}%
    \pgfmathsetmacro\@tmpp{#4}%
    \foreach \@i in {1,...,#2}{%
        \pgfmathsetmacro\@ttmp{div(\@tmp,2)}%
        \pgfmathsetmacro\@ttmpp{div(\@tmpp,2)}%
        \pgfmathsetmacro{\@rr}{int(\@tmp - 2*\@ttmp)}%
        \pgfmathsetmacro{\@rrr}{int(\@tmpp - 2*\@ttmpp)}%
        \pgfmathsetmacro{\@rrrr}{int(\@rr * \@rrr)}%
        \ifthenelse{\@rrrr=1}{
            \global\def\@res{1}
            \breakforeach
        }{}
        \global\let\@tmp=\@ttmp%
        \global\let\@tmpp=\@ttmpp%
    }%

    \def#1{\@res}
}

\newcommand\numtxt[3]{%
    % Converts a bitstring into an array: 0b001 -> {0,0,1}
    % #1 = variable to store the list
    % #2 = length
    % #3 = number (base 10)
    \gdef\store@temp{\@gobble}%
    \pgfmathsetmacro\@tmp{#3}%
    \foreach \@i in {1,...,#2}{%
        \pgfmathsetmacro\@ttmp{div(\@tmp,2)}%
        \pgfmathparse{int(\@tmp - 2*\@ttmp)}%
        \xdef\store@temp{\store@temp,\pgfmathresult}%
        \global\let\@tmp=\@ttmp%
    }%
    \gdef\store@@temp{\@gobble}%
    \foreach \@i in {#2,...,1}{%
        \pgfmathparse{{\store@temp}[\@i-1]}%
        \xdef\store@@temp{\store@@temp,\pgfmathresult}%
    }%
    \let#1=\store@@temp%
}
\newcommand\numnext[3]{%
    % Generates next bitstring with same amount of 1's
    % #1 = variable to store the result
    % #2 = length
    % #3 = number (base 10)

    \gdef\store@temp{\@gobble}%
    \pgfmathsetmacro\@tmp{#3}%
    \foreach \@i in {1,...,#2}{%
        \pgfmathsetmacro\@ttmp{div(\@tmp,2)}%
        \pgfmathparse{int(\@tmp - 2*\@ttmp)}%
        \xdef\store@temp{\store@temp,\pgfmathresult}%
        \global\let\@tmp=\@ttmp%
    }%


    \def\@cnt{0}
    \def\@result{0}
    \def\@ca@r{1}
    \def\@done{0}
    \foreach \@i in {1,...,#2}{%
        \pgfmathsetmacro{\@cc}{{\store@temp}[\@i-1]}%
        \ifthenelse{\@done=1}{
            \ifthenelse{\@cc=1}{
                \pgfmathsetmacro{\@td@}{int(\@result+(\@cc*\@ca@r))}
                \global\let\@result=\@td@
            }{}
       }{
            \ifthenelse{\@cc=1}{
                \pgfmathsetmacro{\@t@mp}{int(\@cnt + 1)}
                \global\let\@cnt=\@t@mp%
            }{
                \ifthenelse{\@cnt>0}{
                    \global\def\@done{1}
                    \pgfmathsetmacro{\@t@}{int(\@result+\@ca@r)}
                    \global\let\@result=\@t@
                    \pgfmathsetmacro{\@tt@}{1}
                    \pgfmathsetmacro{\@ccnt@}{int(\@cnt-1)}
                    \ifthenelse{\@ccnt@>0}{
                    \foreach\@d in {1,...,\@ccnt@}{
                        \pgfmathsetmacro{\@td@}{int(\@result+\@tt@)}
                        \global\let\@result=\@td@
                        \pgfmathsetmacro{\@ttt@}{int(2*\@tt@)}
                        \global\let\@tt@=\@ttt@
                    }
                    }{}
                }{}
            }
        }
        \pgfmathsetmacro{\@ca@r@}{int(2*\@ca@r)}
        \global\let\@ca@r=\@ca@r@
    }

    \let#1=\@result
}
\newcommand\numidx[3]{%
    % Converts a bitstring into an array containing the indices being 1: 0b00110 -> {3,4}
    % #1 = variable to store the list
    % #2 = length
    % #3 = number (base 10)
    \gdef\store@temp{\@gobble}%
    \pgfmathsetmacro\@cnt{0}%
    \pgfmathsetmacro\@tmp{#3}%
    \foreach \@i in {1,...,#2}{%
        \pgfmathsetmacro\@ttmp{div(\@tmp,2)}%
        \pgfmathsetmacro\@r{int(\@tmp - 2*\@ttmp)}%
        \ifthenelse{\@r=1}{%
            \pgfmathsetmacro{\@s}{int(\@cnt+1)}%
            \global\let\@cnt=\@s%
            \pgfmathparse{int(#2-\@i+1)}%
            \xdef\store@temp{\store@temp,\pgfmathresult}%
        }{}%
        \global\let\@tmp=\@ttmp%
    }%
    \xdef\store@temp{\store@temp,}%
    \ifthenelse{\@cnt=0}{%
        \let#1=\empty%
    }{%
        \gdef\store@@temp{\@gobble}%
        \foreach \@i in {\@cnt,...,1}{%
            \pgfmathparse{{\store@temp}[\@i-1]}%
            \xdef\store@@temp{\store@@temp,\pgfmathresult}%
        }%
        \let#1=\store@@temp%
    }%
}

\tikzset{style/.store in=\sty,style=10}
    %   x0/x1/x2 display mode: set/both/string
    %   0y/1y/2y background mode: empty/[fill color]/(unused)


\tikzset{
    % Kneser set visualization
    % #1 length of bitstring
    % #2 bitstring
    pics/kneserset/.style args={#1/#2}{
        code = {
            \pgfmathsetmacro{\bg}{int(div(\sty,10))}
            \pgfmathsetmacro{\mode}{int(\sty - \bg * 10)}

            \ifthenelse{\mode<2}{
                \ifthenelse{#1=0}{}{
                    \numtxt{\arr}{#1}{0b#2}
                }
                \ifthenelse{\bg=0}{
                    \node[circle,minimum size=#1*2pt+16pt]  {};
                    \node[solid,circle,minimum size=#1*2pt+13pt,draw,thick]  {};
                }{
                   \ifthenelse{\mode>0}{
                        \numidx{\res}{#1}{0b#2}
                        \ifx\tikz@fillcolor\pgfutil@empty
%                            \node[fill=white,inner sep=2pt,%minimum width=#1*2pt+16pt,
%                            rounded corners=5pt] (nn) at (0,-#1*1pt-14pt) {$\{\res\}$};
%                            \fill[white]
%                            (nn.east)
%                            [rounded corners=5pt]
%                            -- (nn.north east)
%                            [sharp corners]
%                            -- (tangent cs:node=cc,point={(nn.north east)},solution=1)
%                            -- (tangent cs:node=cc,point={(nn.north west)},solution=2)
%                            [rounded corners=5pt]
%                            -- (nn.north west)
%                            [sharp corners]
%                            -- (nn.west) -- cycle;
                            \node[fill=white,inner sep=2pt,%minimum width=#1*2pt+16pt,
                            rounded corners=5pt] (nn) at (0,-#1*1pt-14pt) {$\{\res\}$};
                        \else
%                            \node[draw=\tikz@fillcolor,fill,inner sep=1.5pt, minimum width=
%                            #1*2pt+16pt, rounded corners=5pt] (nn) at (0,-#1*1pt-14pt) {$\{\res\}$};
%                            \fill
%                            (nn.east)
%                            [rounded corners=5pt]
%                            -- (nn.north east)
%                            [sharp corners]
%                            -- (tangent cs:node=cc,point={(nn.north east)},solution=1)
%                            -- (tangent cs:node=cc,point={(nn.north west)},solution=2)
%                            [rounded corners=5pt]
%                            -- (nn.north west)
%                            [sharp corners]
%                            -- (nn.west) -- cycle;
                            \node[fill,inner sep=2pt,%minimum width=#1*2pt+16pt,
                            rounded corners=5pt] (nn) at (0,-#1*1pt-14pt) {$\{\res\}$};
                        \fi
                    }{}
                    \ifx\tikz@fillcolor\pgfutil@empty
                        \node[circle,minimum size=#1*2pt+16pt,fill=white] (cc) {};
                    \else
                        \node[fill,circle,minimum size=#1*2pt+16pt] (cc) {};
                    \fi
                    \ifthenelse{\bg=1}{
                        \node[solid,circle,minimum size=#1*2pt+14pt,draw,thick]  {};
                    }{}
                    \node[circle,minimum size=#1*2pt+12pt,fill=white] {};
                }
                \ifthenelse{#1<3}{
                    \ifthenelse{#1<2}{
                        \ifthenelse{#1=0}{}{
                            \pgfmathsetmacro{\x}{{\arr}[0]}
                            \ifthenelse{\x=1}{
                                \node[solid,black,draw,fill=black,circle,inner sep=1.5pt] (1) at (0,0) {};
                            }{
                                \node[solid,black,draw,fill=white,circle,inner sep=1.5pt] (1) at (0,0) {};
                            }
                        }
                    }{
                        \foreach \i in {1,...,#1} {
                            \pgfmathsetmacro{\x}{{\arr}[\i-1]}
                            \pgfmathsetmacro{\xx}{pow(-1,\i)*3}
                            \ifthenelse{\x=1}{
                                \node[solid,black,draw,fill=black,circle,inner sep=1.5pt](\i) at (\xx*1pt,0) {};
                            }{
                                \node[solid,black,draw,fill=white,circle,inner sep=1.5pt](\i)at (\xx*1pt,0) {};
                            }
                        }
                    }
                }{
                    \node[regular polygon,regular polygon sides=#1,minimum size=#1*2pt,
                    inner sep=0pt] at (0,0) (poly#1) {};
                    \foreach \i in {1,...,#1} {
                        \pgfmathsetmacro{\x}{{\arr}[\i-1]}
                        \ifthenelse{\x=1}{
                            \node[solid,black,draw,fill=black,circle,inner sep=1.5pt]
                                (\i) at (poly#1.corner \i) {};
                        }{
                            \node[solid,black,draw,fill=white,circle,inner sep=1.5pt] (\i)
                                at (poly#1.corner \i) {};
                        }
                    }
                }
            }{
                \numidx{\res}{#1}{0b#2}
                \ifx\tikz@fillcolor\pgfutil@empty
                    \node[fill=white,inner sep=1.5pt,rounded corners=5pt] at (0,0) {$\{\res\}$};
                \else
                    \node[fill,inner sep=1.5pt,rounded corners=5pt] at (0,0) {$\{\res\}$};
                \fi
            }
        }
    }
}

\def\ldraw(#1)--(#2);{
    \ifx\tikz@fillcolor\pgfutil@empty
        \draw[thick] (#1) -- (#2);
    \else
        \draw[thick,double=black,draw=\tikz@fillcolor] (#1) -- (#2);
    \fi
}

\tikzset{
    % Draw Kneser Graph K(n,k)
    % #1 n
    % #2 k
    % Takes long if n > 7; dies if n > 9
    pics/kneser/.style args={#1/#2}{
        code = {
            \pgfmathsetmacro{\bg}{int(div(\sty,10))}
            \pgfmathsetmacro{\mode}{int(\sty - \bg * 10)}

            \newlinechar=`\^^J
            \pgfmathsetmacro{\diff}{int(#1-#2)}
            \ifthenelse{#2=0}{
                % 0 element subsets ~> boring
                \pic at (0,0) {kneserset=#1/0};
            }{
                \ifthenelse{\diff=0}{
                    \pgfmathsetmacro{\all}{bin(int(pow(2,(#1+1))-1))}
                    \pic at (0,0) {kneserset=#1/\all};
                }{
                    \pgfmathsetmacro{\db}{int(2*#2)}
                    \pgfmathsetmacro{\all}{int(pow(2,(#1+4))-1)}
                    \ifthenelse{\db=#1}{
                        %\pgfmathsetmacro{\tmo}{int(factorial(#1)/(factorial(#1-#2)*factorial(#2)*2)-1)}
                        \pgfmathsetmacro{\tmo}{int(binom(#1,#2)/2-1)}
                        \pgfmathsetmacro{\tmr}{int(ceil(sqrt(\tmo*2)))}

                        \def\strt{0}
                        \foreach\d in {1,...,#2}{
                            \pgfmathsetmacro{\ttt}{int(2*\strt+1)}
                            \global\let\strt=\ttt
                        }

                        \ifthenelse{\tmr<5}{\pgfmathsetmacro\tmr{int(5)}}{}
                        \foreach\i[count=\j from 0] in {0,...,\tmo}{
                            \pgfmathsetmacro{\rem}{int(1+\tmo-(div(\j,\tmr)*\tmr))}
                            \ifthenelse{\rem>\tmr}{\pgfmathsetmacro\rem\tmr}{}

                            \pgfmathsetmacro{\nx}{int(-#1*2-12 + (-div(\tmo,\tmr)/2+div(\i,\tmr))*(#1*2+12+50+#1*2))}
                            \pgfmathsetmacro{\nxx}{int(#1*2+12 + (-div(\tmo,\tmr)/2+div(\i,\tmr))*(#1*2+12+50+#1*2))}

                            \ifthenelse{\mode=1}{\pgfmathsetmacro\dis{35}}
                            {\pgfmathsetmacro\dis{20}}

                            \pgfmathsetmacro{\ny}{(#1*2+\dis)*((\rem-1)/2 - mod(\i,\rem))}

                            \node (1\i) at (\nx*1pt,\ny*1pt) {};
                            \node (2\i) at (\nxx*1pt,\ny*1pt) {};
                            \ldraw(1\i)--(2\i);
                            \draw[thick](1\i)--(2\i);

                            \pgfmathsetmacro{\jj}{bin(int(\strt))}
                            \pgfmathsetmacro{\jd}{bin(int(\all-\strt))}

                            \pic[style=1\mode] at (1\i) {kneserset=#1/\jj};
                            \pic[style=1\mode] at (2\i) {kneserset=#1/\jd};

                            \numnext{\tstrt}{#1}{\strt}
                            \global\let\strt=\tstrt
                        }
                    }{
                    \ifthenelse{\diff=1}{
                        \pgfmathsetmacro{\all}{int(pow(2,(#1+1))-1)}
                        \ifthenelse{\mode=1}{
                            \node[regular polygon,regular polygon sides=#1,minimum size=#1*18pt]
                        }{
                            \node[regular polygon,regular polygon sides=#1,minimum size=#1*14pt]
                        }
                        (p#1) {};
                        \def\tmp{1}
                        \foreach \i[count=\j from 1] in {#1,...,1} {
                            \edef\tp{\tmp}
                            \pgfmathsetmacro{\jd}{bin(int(\all-0b\tmp))}
                            \pic[style=1\mode] at (p#1.corner \i) {kneserset=#1/\jd};
                            \global\edef\tmp{\tp0}
                        }

                    }{
                    \ifthenelse{\db>#1}{
                    % No edges. Just draw the sets in a compact way
                        \pgfmathsetmacro{\tmo}{int(binom(#1,#2))}
                        \def\strt{0}
                        \foreach\d in {1,...,#2}{
                            \pgfmathsetmacro{\ttt}{int(2*\strt+1)}
                            \global\let\strt=\ttt
                        }
                        \pgfmathsetmacro{\tmr}{int(ceil(sqrt(\tmo)))}
                        \ifthenelse{\tmr<1}{\pgfmathsetmacro\tmr{int(1)}}{}

                        \foreach\i[count=\j from 0] in {1,...,\tmo}{
                            \pgfmathsetmacro{\rem}{int(\tmo-(div(\j,\tmr)*\tmr))}
                            \ifthenelse{\rem>\tmr}{\pgfmathsetmacro\rem\tmr}{}

                            \ifthenelse{\mode=1}{\pgfmathsetmacro\dis{35}}
                            {\pgfmathsetmacro\dis{20}}

                            \pgfmathsetmacro{\nx}{int((-div(\tmo,\tmr)/2+div(\j,\tmr))*(#1*2+\dis))}
                            \pgfmathsetmacro{\ny}{(#1*2+\dis)*((\rem-1)/2 - mod(\i,\rem))}

                            \pgfmathsetmacro{\cur}{bin(\strt)}
                            \pic at (\nx*1pt,\ny*1pt) {kneserset=#1/\cur};

                            \numnext{\tstrt}{#1}{\strt}
                            \global\let\strt=\tstrt
                        }
                    }{
                        \ifthenelse{#2=1}{
                            % 1 element subsets ~> draw clique on n vertices

                            \ifthenelse{\mode=1}{
                                \node[regular polygon,regular polygon sides=#1,minimum size=#1*18pt]
                            }{
                                \node[regular polygon,regular polygon sides=#1,minimum size=#1*14pt]
                            }
                            (p#1) {};
                            \foreach \i in {#1,...,1} { \foreach \j in {#1,...,1} {
                                    \ifthenelse{\i=\j}{}{
                                        \ldraw(p#1.corner \i)--(p#1.corner \j);
                                    }
                            }}
                            \foreach \i in {#1,...,1} { \foreach \j in {#1,...,1} {
                                    \ifthenelse{\i=\j}{}{
                                        \draw[thick](p#1.corner \i)--(p#1.corner \j);
                                    }
                            }}

                            \def\tmp{1}
                            \foreach \i[count=\j from 1] in {#1,...,1} {
                                \edef\tp{\tmp}
                                \pic[style=1\mode] at (p#1.corner \i) {kneserset=#1/\tmp};
                                \global\edef\tmp{\tp0}
                            }
                        }{
                        \ifthenelse{#2=2}{
                            \pgfmathsetmacro{\nmk}{int(#1 - #2)}
                            \pgfmathsetmacro{\nodd}{isodd(#1)}
                            \pgfmathsetmacro{\ncnt}{int(binom(#1,#2))}
                            \pgfmathsetmacro{\ecnt}{int(\ncnt*binom(\nmk,#2)/2)}

                            \pgfmathsetmacro{\nrem}{int(\ncnt)}
                            \pgfmathsetmacro{\nid}{int(0)}

                            \pgfmathsetmacro{\mxn}{int(pow(2,#1)-1)}

                            \pgfmathsetmacro{\rot}{int(isodd(#1) ? 0:-180/#1)}

                            \pgfmathsetmacro{\nly}{int(isodd(#1) ? div(\ncnt,#1) : 1 + div(\ncnt,#1))}
                            \pgfmathsetmacro{\lid}{int(1)}
                            \pgfmathsetmacro{\nexl}{int(1 + max(1,div(\nly,2)))}

                            \gdef\seen{\@gobble}
                            \gdef\poli{\@gobble}
                            \gdef\polc{\@gobble}
                            \ifthenelse{\nodd=1}{
                                \foreach\lid in {1,...,\nly}{
                                    \pgfmathsetmacro{\ctm}{int(1 + pow(2,1+2*(\lid-1)))}
                                    \foreach \j in {#1,...,1} {
                                        \pgfmathsetmacro{\idx}{int(mod(\j + div(#1,2) - \lid, #1)+1)}
                                        \pgfmathsetmacro{\curset}{bin(\ctm)}
                                        \xdef\seen{\seen,\curset}
                                        \xdef\poli{\poli,\lid}
                                        \xdef\polc{\polc,\idx}

                                        \pgfmathsetmacro{\ctmm}{int(2*\ctm > \mxn ? mod(2*\ctm + 1,\mxn+1) : 2*\ctm)}
                                        \global\let\ctm=\ctmm
                                    }
                                }

                                \foreach\ld in {1,...,\nexl}{
                                    \pgfmathsetmacro{\doub}{int((#1<\nrem) && \ld>1)}
                                    \ifthenelse{\doub=1}{
                                        \pgfmathsetmacro{\tm}{int(\nrem - #1)}
                                        \global\let\nrem=\tm
                                        \pgfmathsetmacro{\rott}{int(-90/#1)}
                                        \ifthenelse{\mode=1}{
                                            \node[regular polygon,regular polygon sides=#1,
                                            minimum size=#1*\ld*19pt-#1*3pt,
                                            rotate=\rot+\rott](p\lid){};
                                        }{
                                            \node[regular polygon, regular polygon sides=#1,
                                            minimum size=#1*\ld*15pt-#1*3pt,
                                            rotate=\rot+\rott](p\lid){};
                                        }
                                        \pgfmathsetmacro{\tm}{int(\lid+1)}
                                        \global\let\lid=\tm
                                        \ifthenelse{\mode=1}{
                                            \node[regular polygon,regular polygon sides=#1,
                                            minimum size=#1*\ld*20pt-#1*3pt,
                                            rotate=\rot-\rott](p\lid){};
                                        }{
                                            \node[regular polygon, regular polygon sides=#1,
                                            minimum size=#1*\ld*16pt-#1*3pt,
                                            rotate=\rot-\rott](p\lid){};
                                        }
                                        \pgfmathsetmacro{\tm}{int(\lid+1)}
                                        \global\let\lid=\tm
                                        \pgfmathsetmacro{\tm}{int(\nrem - #1)}
                                        \global\let\nrem=\tm
                                    }{
                                        \pgfmathsetmacro{\rott}{int(#1>5&&\lid==1?-#1/2:0)}
                                        \ifthenelse{\mode=1}{
                                            \node[regular polygon,regular polygon sides=#1,
                                            minimum size=#1*\ld*19pt-#1*3pt,
                                            rotate=\rott](p\lid){};
                                        }{
                                            \node[regular polygon, regular polygon sides=#1,
                                            minimum size=#1*\ld*15pt-#1*3pt,
                                            rotate=\rott](p\lid){};
                                        }
                                        \pgfmathsetmacro{\tm}{int(\lid+1)}
                                        \global\let\lid=\tm
                                        \pgfmathsetmacro{\tm}{int(\nrem - #1)}
                                        \global\let\nrem=\tm
                                    }
                                }
                            }{
                                \pgfmathsetmacro{\rotsh}{iseven(#1/2)}
                                \foreach\lid in {1,...,\nly}{
                                    \pgfmathsetmacro{\curn}{int((\lid==ceil(#1/4)&&not(\rotsh))
                                    || (\lid==\nly && \rotsh)?#1/2:#1)}
                                    \pgfmathsetmacro{\bnd}{int(int(\curn!=#1 ? 1 : \curn)}
                                    \pgfmathsetmacro{\bend}{int(int(\curn!=#1 ? \curn :1)}
                                    \pgfmathsetmacro{\ctm}{int(1 + pow(2,(\lid>ceil(#1/4) ? 0: 1)
                                        +2*(\lid>ceil(#1/4)?\lid-div(#1,4):\lid-1)))}
                                    \foreach \j in {\bnd,...,\bend} {
                                        \ifthenelse{#1=6}{
                                            \pgfmathsetmacro{\idx}{int(mod(\j-2+\lid-(\lid==2), \curn)+1)}
                                        }{}
                                        \ifthenelse{#1=8}{
                                            \pgfmathsetmacro{\idx}{int(mod(\j-1*(\lid==1)+6*(\lid>1), \curn)+1)}
                                        }{}

                                        \pgfmathsetmacro{\curset}{bin(\ctm)}
                                        \xdef\seen{\seen,\curset}
                                        \xdef\poli{\poli,\lid}
                                        \xdef\polc{\polc,\idx}
                                        \pgfmathsetmacro{\ctmm}{int(2*\ctm > \mxn ? mod(2*\ctm + 1,\mxn+1) : 2*\ctm)}
                                        \global\let\ctm=\ctmm
                                    }
                                }
                                \foreach\lid in {1,...,\nly}{
                                    \pgfmathsetmacro{\level}{int(\lid>ceil(#1/4)?1+\lid-ceil(#1/4):\lid)}
                                    \pgfmathsetmacro{\level}{#1==6&&\lid==3?1.8:\level}
                                    \pgfmathsetmacro{\level}{#1==6&&\lid==1?1.07:\level}
                                    \pgfmathsetmacro{\curn}{int((\lid==ceil(#1/4)&&not(\rotsh))
                                        || (\lid==\nly && \rotsh)?#1/2:#1)}
                                    \pgfmathsetmacro{\rott}{int((\lid>ceil(#1/4)&&\rotsh?180/#1:0)
                                    + (not(\rotsh)&&\lid>1?-180/#1:0)
                                    + (not(\rotsh)&&\curn!=#1?360/#1:0))}
                                    \pgfmathsetmacro{\rott}{int(\lid==1?\rott-#1/2:\rott)}

                                    \ifthenelse{\mode=1}{
                                        \node[regular polygon,regular polygon sides=\curn,
                                        minimum size=#1*\level*19pt-#1*3pt,
                                        rotate=\rot+\rott](p\lid){};
                                    }{
                                        \node[regular polygon, regular polygon sides=\curn,
                                        minimum size=#1*\level*15pt-#1*3pt,
                                        rotate=\rot+\rott](p\lid){};
                                    }

                                }
                            }
                            \gdef\prsx{\@gobble}
                            \gdef\prsy{\@gobble}
                            \foreach\i in {1,...,\ncnt}{
                                \pgfmathsetmacro{\curseta}{{\seen}[\i-1])}
                                \pgfmathsetmacro{\cplia}{{\poli}[\i-1])}
                                \pgfmathsetmacro{\cplca}{{\polc}[\i-1])}
                                \foreach\j in {\i,...,\ncnt}{
                                    \pgfmathsetmacro{\cursetb}{{\seen}[\j-1])}
                                    \pgfmathsetmacro{\cplib}{{\poli}[\j-1])}
                                    \pgfmathsetmacro{\cplcb}{{\polc}[\j-1])}
                                    \intersect{\inte}{#1}{0b\curseta}{0b\cursetb}
                                    \ifthenelse{\inte=0}{
                                        \ldraw(p\cplia.corner \cplca)--(p\cplib.corner \cplcb);
                                        \xdef\prsx{\prsx,\i}
                                        \xdef\prsy{\prsy,\j}
                                    }{}
                                }
                            }
                            \foreach\k in {1,...,\ecnt}{
                                \pgfmathsetmacro{\i}{{\prsx}[\k-1])}
                                \pgfmathsetmacro{\j}{{\prsy}[\k-1])}
                                \pgfmathsetmacro{\cplia}{{\poli}[\i-1])}
                                \pgfmathsetmacro{\cplca}{{\polc}[\i-1])}
                                \pgfmathsetmacro{\cplib}{{\poli}[\j-1])}
                                \pgfmathsetmacro{\cplcb}{{\polc}[\j-1])}
                                \draw[thick](p\cplia.corner \cplca)--(p\cplib.corner \cplcb);
                            }


                            \foreach\i in {1,...,\ncnt}{
                                \pgfmathsetmacro{\pli}{{\poli}[\i-1])}
                                \pgfmathsetmacro{\plc}{{\polc}[\i-1])}
                                \pgfmathsetmacro{\curset}{{\seen}[\i-1])}
                                \pic[style=1\mode] at (p\pli.corner \plc){kneserset=#1/\curset};
                            }
                            }{
                                \ifthenelse{#1>2}{
                                    \pgfmathsetmacro{\tmo}{int(binom(#1,#2))}
                                    \ifthenelse{\mode=1}{
                                        \node[regular polygon,regular polygon sides=\tmo,minimum size=\tmo*20pt](o#1){};
                                    }{
                                        \node[regular polygon,regular polygon sides=\tmo,minimum size=\tmo*10pt](o#1){};
                                    }
                                    \def\strt{0}
                                    \foreach\d in {1,...,#2}{
                                        \pgfmathsetmacro{\ttt}{int(2*\strt+1)}
                                        \global\let\strt=\ttt
                                    }
                                    \gdef\seen{\@gobble}
                                    \gdef\prsx{\@gobble}
                                    \gdef\prsy{\@gobble}
                                    \def\cnt{0}
                                    \foreach\i[count=\j from 0] in {1,...,\tmo}{
                                        \ifthenelse{\j>0}{
                                            \foreach\g in {1,...,\j}{
                                                \pgfmathsetmacro{\cur}{int({\seen}[\g-1])}
                                                \intersect{\inte}{#1}{\cur}{\strt}
                                                \ifthenelse{\inte=0}{
                                                    \ldraw(o#1.corner \i)--(o#1.corner \g);
                                                    \xdef\prsx{\prsx,\i}
                                                    \xdef\prsy{\prsy,\g}
                                                    \pgfmathsetmacro{\gnr}{int(\cnt+1)}
                                                    \global\let\cnt=\gnr
                                                }{}
                                            }
                                        }{}

                                \xdef\seen{\seen,\strt}
                                \numnext{\tstrt}{#1}{\strt}
                                \global\let\strt=\tstrt
                            }
                            \ifthenelse{\cnt>0}{
                            \foreach\i in {1,...,\cnt}{
                                \pgfmathsetmacro{\curx}{{\prsx}[\i-1])}
                                \pgfmathsetmacro{\cury}{{\prsy}[\i-1])}
                                \draw[thick](o#1.corner \curx)--(o#1.corner \cury);
                            }}{}
                            \foreach\i[count=\j from 0] in {1,...,\tmo}{
                                \pgfmathsetmacro{\cur}{bin({\seen}[\j])}
                                \pic[style=1\mode] at (o#1.corner \i) {kneserset=#1/\cur};
                            }
                        }{}
                    }
                        }
                    }}}
                }
            }
        }
    }
}

\makeatother

